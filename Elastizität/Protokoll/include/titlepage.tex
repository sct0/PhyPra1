% coordinates for background border
\newcommand{\diameter}{20}
\newcommand{\xone}{-15}
\newcommand{\xtwo}{160}
\newcommand{\yone}{15}
\newcommand{\ytwo}{-253}

\newcommand{\hoehea}{55}
\newcommand{\hoeheb}{55}




\begin{titlepage}
    % background border
    \begin{tikzpicture}[overlay]
    \draw[color=gray]  
            (\xone mm, \yone mm)
      -- (\xtwo mm, \yone mm)
    arc (90:0:\diameter pt) 
      -- (\xtwo mm + \diameter pt , \ytwo mm) 
        -- (\xone mm + \diameter pt , \ytwo mm)
    arc (270:180:\diameter pt)
        -- (\xone mm, \yone mm);
    \end{tikzpicture}
    
    % KIT logo
    \begin{textblock}{10}[0,0](4.5,2.5)
        \includegraphics[width=.25\textwidth]{include/kitlogo.pdf}
    \end{textblock}
    \changefont{phv}{m}{n}    % helvetica
    \begin{textblock}{10}[0,0](5.5,2.2)
        \begin{flushright}
            \Large FAKULTÄT FÜR PHYSIK\\Praktikum Klassische Physik
        \end{flushright}
    \end{textblock}
    
    \begin{textblock}{10}[0,0](4.2,3.1)
        \begin{tikzpicture}[overlay]
        \draw[color=gray]
            (\xone mm + 5 mm, -8 mm)
         -- (\xtwo mm + \diameter pt - 5 mm, -8 mm);
        \end{tikzpicture}
    \end{textblock}
    
    \Large
    % Zeile 1
    \begin{textblock}{12}[0,0](3.58,4)
        \mytextfield{Prak.}{\praktikum}{0.9cm}{17pt}
                    {P1/P2}{2}{Praktikum}
    \end{textblock}
    \begin{textblock}{12}[0,0](5.53,4)
        \mytextfield{Semester}{\semester}{2.6cm}{17pt}
        {z.B. \glqq WS14/15\grqq\ oder \glqq SS15\grqq}{0}{Semester}
    \end{textblock}
    \begin{textblock}{12}[0,0](9.53,4)
        \mytextfield{Wochentag}{\wochentag}{1.3cm}{17pt}
                    {Mo/Di/Mi/Do}{2}{Wochentag}
    \end{textblock}
    \begin{textblock}{12}[0,0](12.88,4)
       \mytextfield{Gruppennr.}{\gruppennr}{1.06cm}{17pt}
                   {\#\#}{2}{Gruppennummer}
    \end{textblock}
    
    % Zeile 2
    \begin{textblock}{12}[0,0](3.58,4.55)
        \mytextfield{Name}{\nachnamea}{6cm}{17pt}
                    {}{0}{Name1}
    \end{textblock}
    \begin{textblock}{12}[0,0](9.53,4.55)
        \mytextfield{Vorname}{\vornamea}{6cm}{17pt}
                    {}{0}{Vorname1}
    \end{textblock}
    
    % Zeile 3
    \begin{textblock}{12}[0,0](3.58,5.1)
        \mytextfield{Name}{\nachnameb}{6cm}{17pt}
                    {}{0}{Name2}
    \end{textblock}
    \begin{textblock}{12}[0,0](9.53,5.1)
        \mytextfield{Vorname}{\vornameb}{6cm}{17pt}
                    {}{0}{Vorname2}
    \end{textblock}
    
    % Zeile 4
    \begin{textblock}{12}[0,0](3.64,5.65)
       \normalsize\mytextfield{Emailadresse(n)}{\emailadressen}{13.1cm}{10pt}
                              {Optional}{0}{Emailadressen}
    \end{textblock}
    
    % Zeile 5
    \begin{textblock}{12}[0,0](3.58,6.2)
        \mytextfield{Versuch}{\versuch\ (\praktikum-\versuchsnr)}{9.45cm}{14pt}
                    {z.B. \glqq Galvanometer (P1-13)\grqq\ oder \glqq %
                     Mikrowellenoptik (P2-15)\grqq}{0}{Versuch}
    \end{textblock}
    \begin{textblock}{12}[0,0](12.58,6.2)
       \mytextfield{Fehlerrech.}{\fehlerrechnung}{1.46cm}{17pt}
                   {Ja/Nein}{4}{Fehlerrechnung}
    \end{textblock}
    
    % Zeile 6
    \begin{textblock}{12}[0,0](3.58,6.75)
        \mytextfield{Betreuer}{\betreuer}{7cm}{17pt}{}{0}{Betreuer}
    \end{textblock}
    \begin{textblock}{12}[0,0](10.82,6.75)
        \mytextfield{Durchgeführt am}{\durchgefuehrt}{2.53cm}{17pt}
                    {TT.MM.JJ}{8}{Durchfuehrung}
    \end{textblock}
    
    % Querstrich
    \begin{textblock}{20}[0,0](0,7.1)\tiny\centering
        Wird vom Betreuer ausgefüllt.
    \end{textblock}
    \begin{tikzpicture}[overlay]
    \draw[color=gray]
        (\xone mm + 5 mm, -78 mm)
     -- (\xtwo mm + \diameter pt - 5 mm, -78 mm);
    \end{tikzpicture}
    
    % Zeile 1
    \begin{textblock}{12}[0,0](3.58,8)
        \myTtextfield{1. Abgabe am}{}{2.5cm}{17pt}
                     {}
    \end{textblock}
    \begin{textblock}{20}[0,0](8.3,8)
        \myTtextfield{Rückgabe am}{}{2.5cm}{17pt}
                     {}
    \end{textblock}
    
    % Block 1
    \begin{tikzpicture}[overlay]
    \draw[color=gray]  
        (\xone mm + 10 mm, -85.5 mm)
     -- (\xtwo mm + \diameter pt - 10 mm, -85.5 mm)
     -- (\xtwo mm + \diameter pt - 10 mm, -85.5 mm - \hoehea mm)
     -- (\xone mm + 10 mm, -85.5 mm - \hoehea mm)
     -- (\xone mm + 10 mm, -85.5 mm);
    \end{tikzpicture}
    
    \begin{textblock}{20}[0,0](4,8.57)
        Begründung:
    \end{textblock}
    
    % Zeile 2
    \begin{textblock}{12}[0,0](3.58,11.85)
        \myTtextfield{2. Abgabe am}{}{2.5cm}{17pt}
                     {}
    \end{textblock}
    
    % Block 2
    \begin{tikzpicture}[overlay]
    \draw[color=gray]  
        (\xone mm + 10 mm, -167 mm)
     -- (\xtwo mm + \diameter pt - 10 mm, -167 mm)
     -- (\xtwo mm + \diameter pt - 10 mm, -167 mm - \hoehea mm)
     -- (\xone mm + 10 mm, -167 mm - \hoehea mm)
     -- (\xone mm + 10 mm, -167 mm);
    \end{tikzpicture}
    \begin{textblock}{12}[0,0](4.25,12.24)
        {Ergebnis:~~~~+~~~/~~~0~~~/~~~-}
    \end{textblock}
    \begin{textblock}{12}[0,0](9.1,12.24)
        {Fehlerrechnung:~~~Ja~~~/~~~Nein}
    \end{textblock}
    \begin{textblock}{12}[0,0](4.05,12.9)
        \myTtextfield{Datum}{}{2.5cm}{17pt}
                     {}
    \end{textblock}
    \begin{textblock}{12}[0,0](8.9,12.9)
        \myTtextfield{Handzeichen}{}{5.5cm}{17pt}
                     {}
    \end{textblock}
    \begin{textblock}{12}[0,0](4,13.4)\Large
        {Bemerkungen:}
    \end{textblock}
    
    
    
    % lowest text blocks concerning the KIT
    \begin{textblock}{10}[0,0](4,16.8)
        \tiny{KIT -- Universität des Landes Baden-Württemberg und nationales %
              Forschungszentrum in der Helmholtz-Gemeinschaft}
    \end{textblock}
    \begin{textblock}{10}[0,0](14,16.75)
        \large{\textbf{www.kit.edu}}
    \end{textblock}
\end{titlepage}
