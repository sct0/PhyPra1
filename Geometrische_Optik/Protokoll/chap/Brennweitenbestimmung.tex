\section{Einfache Bestimmung der Brennweite einer Linse}
\begin{table}[]
    \centering
    \begin{tabular}{|c|c|}
    	\hline
    	Messung Nr. & Pos. Linse (cm) \\
    	\hline
    	1 & 178.5 \\
    	\hline
    	2 & 178.4 \\
    	\hline
    	3 & 178.4 \\
    	\hline
    	4 & 178.4 \\
    	\hline
    	5 & 178.5 \\
    	\hline
    	6 & 178.4 \\
    	\hline
    	7 & 178.5 \\
    	\hline
    	8 & 178.3 \\
    	\hline
    	9 & 178.5 \\
    	\hline
    	10 & 178.5 \\
    	\hline
    
    \end{tabular}
    \caption{Daten: Einfache Bestimmung der Brennweite einer Linse}
    \label{tab:Daten1}
\end{table}


In diesem Versuchsteil soll die Brennweite einer Linse nur mit Hilfe eines Maßstabs und eines Schirms kontrolliert werden. Dazu wird sowohl die Linse als auch ein Schirm auf einer optischen Bank montiert. Um den Fehler bei dieser Messung zu minimieren wird der Abstand zwischen Lichtquelle groß gewählt bzw. gegebenenfalls ein Kondensor nach der Lichtquelle zwischengeschaltet. Für die Messung wird nun die Linse solange verschoben, bis ein möglichst kleiner Lichtpunkt auf dem Schirm entsteht, der Abstand zwischen Schirm und der Linse ist somit die Brennweite. Für den Versuch wird eine Linse mit einer angegebenen Brennweite von $F = 15 cm$ gewählt, dabei wird der Schirm auf der 195cm-Marke der optischen Bank fixiert. Es ergeben sich die Werte in der Tabelle \ref{tab:Daten1}. Somit ergibt sich ein Mittelwert von $16.5600$ mit einer Standardabweichung von $0.066cm$, also: $16.5600cm \pm 0.066cm$. Was einer Abweichung von ca. $9 \% $ zum angegebenen Wert entspricht. Diese Abweichung kommt wahrscheinlich durch die ungenaue Messmethode zustande. Faktoren, die die Genauigkeit der Messung beeinflussen sind etwa, dass die Messung nicht mit monochromatischem Licht durchgeführt wurde und es somit zu einem anderen Ergebnis kommen könnte, da die Brechzahl auch von der Wellenlänge des Lichts abhängt. Allerdings wurden zehn dicht aneinander liegende Werte ermittelt, was eventuell auch auf eine nicht korrekte Angabe der Brennweite auf der Linse hindeuten könnte.
\\
SKIZZE

\section{Brennweitenbestimmung einer Linse mit dem Besselverfahren}
\section{Brennweitenbestimmung eines Zweilinsensystems mit dem Abbéschen Verfahren}