\section{Grundlagen der Aeromechanik}

Die Aeromechanik beschäftigt sich mit dem Verhalten von Gasen und deren mechanische Auswirkungen auf umströmte Körper.
Grundlegend zur Beschreibung dieser Vorgänge sind die Zusammenhänge zwischen \\
dem statischem und dynamischem Druck (auch Staudruck genannt)\\
dem Strömungswiderstand des umströmten Körpers \\
dessen Fläche\\
und der Geschwindigkeit des Gases.

Die Beziehung für die Kraft, die ein Gas auf einen Körper ausübt, ergibt sich aus Messungen ähnlich denen, die wir später genauer untersuchen werden und lautet

\begin{equation} \label{Kraft}
    F = c_w \cdot \frac{\rho}{2}u^2\cdot A
\end{equation}


    
\begin{equation} \label{Kontinuitaetsgleichung}
    A_1u_1\rho_1 = A_2u_2\rho_2
\end{equation}

\begin{equation} \label{Kraft}
    p + \frac{\rho}{2}u^2 = p_0
\end{equation}