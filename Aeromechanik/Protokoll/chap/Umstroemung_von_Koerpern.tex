\section{Rücktrieb und Stirnfläche}

In diesem Versuch soll der Rücktrieb in Abhängigkeit mit der Stirnfläche gemessen werden. Um so einen Teil der Formel für die Kraft in Formel \ref{Kraft} experimentell herzuleiten.

\subsection{Messaufbau}

Um diese Messung zu realisieren werden drei verschieden Große Kreisscheiben in den Luftstrom gebracht und die rücktreibenden Kräfte mit einem Kraftmesser gemessen.

\subsection{Auswertung}

Wir erhalten mit der Messung die Werte in Tabelle \ref{tab:Aufgabe2.1}. Die Werte für den Rücktrieb pro Fläche sind ähnlich, müssten aber unter idealen Bedingungen in der Theorie gleich sein. Allerdings lässt sich diese Abweichung einerseits auf die verschwindend kleine Messreihe zurückführen und auch auf die ungenaue Messmethode, da zum Beispiel Reibung eine große Rolle spielt. Was aber einen wesentlich größeren Effekt hat und auch die gemessene Werte erklären würde, ist die Tatsache, dass es deutlich mehr Verwirbelungen pro Fläche bei kleineren Scheiben im Vergleich zu größeren gibt, was wiederum zu einem größeren Wert für den Rücktrieb pro Fläche führt.

\begin{table}[h]
    \caption{Rücktreibende Kraft für verschieden große Kreisscheiben}
    \centering
    \begin{tabular}{c c c}
    \hline
    Radius Kreisscheibe     & Rücktrieb  & Rücktrieb / Fläche\\
    \hline
    \SI{4}{\centi\metre}     & \SI{0.53}{\newton} & \SI{105.44}{\newton\per\square\metre}\\[5pt]
    \SI{2.8}{\centi\metre}  &   \SI{0.31}{\newton} & \SI{125.86}{\newton\per\square\metre} \\[5pt]
    \SI{2}{\centi\metre}    &   \SI{0.17}{\newton} & \SI{135.28}{\newton\per\square\metre} \\[5pt]
    \hline
    \end{tabular}
    \label{tab:Aufgabe2.1}
\end{table}

\section{Rücktrieb und Strömungsgeschwindigkeit}
\subsection{Messaufbau}

\section{Rücktrieb und Körperform}
\subsection{Messaufbau}

\section{Modellauto}
\subsection{Messaufbau}