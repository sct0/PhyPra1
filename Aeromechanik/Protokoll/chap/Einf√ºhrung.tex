\section{Grundlagen der Aeromechanik}

Die Aeromechanik beschäftigt sich mit dem Verhalten von Gasen und deren mechanische Auswirkungen auf umströmte Körper.
Grundlegend zur Beschreibung dieser Vorgänge sind die Zusammenhänge zwischen \\
dem statischem und dynamischem Druck (auch Staudruck genannt),\\
dem Strömungswiderstand des umströmten Körpers, \\
dessen Fläche\\
und der Geschwindigkeit des Gases.

Die Beziehung für die Kraft, die ein Gas auf einen Körper ausübt, ergibt sich aus Messungen ähnlich denen, die wir später genauer untersuchen werden und lautet

\begin{equation} \label{Kraft}
    F = c_w \cdot \frac{\rho}{2}u^2\cdot A
\end{equation}

Die Kontinuitätsgleichung enthält eine weitere wichtige Beziehung und gibt an, wie sich die Strömungsgeschwindigkeit bei Änderung des Querschnitts verhält.
    
\begin{equation} \label{Kontinuitaetsgleichung}
    A_1u_1\rho_1 = A_2u_2\rho_2
\end{equation}

Die Bernoullische Gleichung enthält eine der wichtigsten Beziehungen der Aeoromechanik, zwischen statischem, dynamischen und Gesamtdruck. Wobei letzteres oft als konstant angenommen wird. Mit dieser Gleichung alleine lassen sich schon viele Phänomene der Aeromechanik beschreiben und wird in diesem Versuch bei vielen Versuchen Verwendung finden.

\begin{equation} \label{Bernoullische Gleichung}
    p + \frac{\rho}{2}u^2 = p_0
\end{equation}