\section{Aufbau und Durchführung}

Bei dieser Versuchsreihe wird die Lichtgeschwindigkeit über die Zeitdifferenz zwischen dem Senden und dem Empfangen von Lichtimpulsen errechnet.
Die gemessen wird die Zeitdifferenz in Abgängigkeit von der Entfernung zwischen Sender und Empfänger.
Der Aufbau besteht aus einer Metallschiene, auf der eine horizontal verschiebbare, rote LED Lampe angebracht ist.
An einem Ende der Schiene steht ein Oszilloskop mit einem Lichtempfänger, auf den die LED gerichtet ist.
Das Oszilloskop ist an einen Computer angeschlossen, die Messung des Phasenversatzes geschieht dann via Software.
Zusätzlich ist an dem Halter der LED ein Lasermessgerät angebracht, an dem der Abstand zwischen Lampe und Oszilloskop abgelesen wird.

Bei der Durchführung wurde die LED in verschiedene Abstände zum Oszilloskop geschoben und dann in der Software die zeitliche Verschiebung zwischen Sender (LED) und Lichtempfänger bestimmt.
Der Versuch wurde mit drei verschiedenen Medien zwischen LED und Oszilloskop durchgeführt: Luft, Wasser und Plexiglasstangen in zwei signifikant verschiedenen Längen

Aufgrund der technischen Einschränkungen des Oszilloskopes ist für diesen Versuch noch der Vergrößerungfaktor der Zeit zu berücksichtigen.
Dieser ergibt sich aus den in der Aufgabenstellungen angebenen Werten von $\omega = 2 \cdot \pi \cdot \SI{60}{MHz}$ und $\Omega = 2 \cdot \pi \cdot \SI{59,9}{MHz}$ zu:

\begin{equation}
    V = \frac{\omega}{\omega - \Omega} = \frac{\SI{60}{MHz}}{\SI{100}{KHz}} = \SI{600}{}
\end{equation}

\section{Justierung und Eichung}

Die Justierung und Eichung des Versuchsaufbaus wurde im Rahmen des Versuches nach Vorgabe des Betreuers nicht durchgeführt.

\section{Messung der Lichtgeschwindigkeit in Luft}

- Hinweis: Bei den Messwerten der Distanzen sind Nachhinein Ungereimtheiten aufgefallen, daher wurden aus zeitgründen ersatzweise Messwerte vom Betreuer zu Verfügung gestellt, mit denen im Folgenden gehandhabt wird. -

Der Versuch wird zunächst ohne spezielles Medium, einfach nur in Luft durchgeführt.

Es ergeben sich folgende Messwerte:

\begin{table}[h!]
    \begin{center}
        \caption{Messungen der Phasenverschiebung für verschiedene Abstände in Luft}
        \begin{tabular}{cc}
            \hline
            Abstand in $\SI{}{cm}$ & Zeitdifferenz in $\SI{}{\mu s}$ \\
            \hline
            $\SI{260,1}{}$    & $\SI{4,831}{}$ \\
            $\SI{239,3}{}$    & $\SI{4,399}{}$ \\
            $\SI{220,6}{}$    & $\SI{4,008}{}$ \\
            $\SI{200,7}{}$    & $\SI{3,643}{}$ \\
            $\SI{181,3}{}$    & $\SI{3,261}{}$ \\
            $\SI{160,0}{}$    & $\SI{2,837}{}$ \\
            $\SI{140,5}{}$    & $\SI{2,430}{}$ \\
            $\SI{119,9}{}$    & $\SI{2,040}{}$ \\
            $\SI{101,4}{}$    & $\SI{1,682}{}$ \\
            $\SI{80,8}{}$     & $\SI{1,275}{}$ \\
            $\SI{61,1}{}$     & $\SI{0,9255}{}$ \\
            $\SI{40,2}{}$     & $\SI{0,4779}{}$ \\
            $\SI{20,7}{}$     & $\SI{0,1041}{}$ \\
            \hline
            \label{tab:Messwerte-Zeitdiffernz-Abstand}
        \end{tabular}
    \end{center}
\end{table}

Die Messreihe wird über lineare Regression ausgewertet:



Der relevante Wert ist hier die Steigung $m = 1.967110{\frac{m}{\mu s}} \pm 0.003719{\frac{m}{\mu s}}$, dieser wird noch mit dem Vergrößerungsfaktor korrigiert und es ergibt sich:

\begin{equation}
    c_0  = m = \SI{0,00327}{\frac{m}{\mu s}} \pm 0,0000062{\frac{m}{\mu s}}\,.
\end{equation}

Dieser Wert weicht erstaunlich stark vom Literaturwert von $\SI{3e8}{\frac{m}{s}}$ ab.

\subsection{Fehlerbetrachtung}

Der statistische Fehler ergibt sich aus der Regression und beträgt $\pm 0,0000062{\frac{m}{\mu s}}$.

Der systematische Fehler setzt sich aus der Fehlerbehaftung der Streckenmessung und der Fehlerbehaftung des abgelesenen Wertes der Phasenverschiebung.
Da ein Lasermessgerät verwendet wurde, setzen wir den Fehler hier sehr klein bei $\SI{0,00005}{m}$ an.

Den Wert der Phasenverschiebung behaften wir mit $\SI{0,01}{\mu s}$, da der Wert in der Software per Hand- und Augenmaß eingestellt wurde.
Eine weitere Fehlerquelle ist hier, dass der Lichtstrahl mit zunehmenden Abstand diffuser wird und dadurch das Signal geschwächt wurde.

Der Vergrößerungsfaktor wird unbehaftet belassen, da die zu Grunde liegenden Werte ohne Fehler vorgegeben waren.

Aus diesem Grund würde es sich anbieten, das Wertepaar mit dem kleinsten Abstand zur Fehlerberechnung zu verwenden, allerdings fällt bei kleineren Strecken der Fehler auf dem Abstand mehr ins Gewicht.
Daher haben wir uns für das mittlere Wertepaar ($\SI{119,9}{}$,$\SI{2,040}{}$) entschieden.

Über Gauß'sche Fehlerfortpflanzung ergibt sich ein Fehler von $\SI{0,0005}{\frac{m}{s}}$.

Mit Fehlern beträgt die errechnete Lichtgeschwindigkeit somit:

$c = \SI{0,00327}{\frac{m}{\mu s}} \pm \SI{0,0000062}{\frac{m}{\mu s}}{\frac{m}{s}} \pm \SI{0,0005}{\frac{mu}{s}}$


\section{Bestimmung der Brechzahl in Wasser und Plexiglas}

Bei dieser Messung werden unterschiedliche Medien zwischen Sender und Empfänger gestellt und jeweils eine Messung ohne und mit Medium beim gleichen Abstand betrachtet, wobei die Lichtgeschwindigkeit in Luft hier mit der Lichtgeschwindigkeit im Vakuum gleichgesetzt wird.

Um die Brechzahl zu errechnen, muss man zunächst die Differenz der Zeiten aus den Messungen mit und ohne Material bilden:

\begin{equation}
    t_o' = \frac{s_l}{c_l}\,.
\end{equation}

\begin{equation}
    t_m' = \frac{s_l - s_m}{c_l} + \frac{s_m}{c_m}\,.
\end{equation}

\begin{equation}
   => t_m' - t_o' = \frac{s_m}{c_m} - \frac{s_m}{c_l}\,.
\end{equation}

Nun muss man noch die Vergrößerung berücksichtigen:

\begin{equation}
    \frac{1}{\SI{600}{} \cdot s_m} \cdot (t_m - t_o) = \frac{1}{c_m} - \frac{1}{c_l}\,.
\end{equation}

Und erhält nach Umformung:

\begin{equation}
    n_m = \frac{c_l}{c_m} = \frac{c_l}{c_m} = \frac{(t_m - t_o) \cdot c_l}{\SI{600}{} \cdot s_m} + 1\,.
\end{equation}

Für die Lichtgeschwindigkeit in Luft wird jeweils der Literaturwert von $\SI{3e8}{\frac{m}{s}}$ verwendet.

\subsection{Wasser}

\begin{table}[h!]
    \begin{center}
        \caption{Messungen der Phasenverschiebung für verschiedene Abstände mit Wasser, 1m}
        \begin{tabular}{cc}
            \hline
            Abstand in $\SI{}{cm}$ & Zeitdifferenz in $\SI{}{\mu s}$ \\
            \hline
            $\SI{260,1}{}$    & $\SI{5,562}{}$ \\
            $\SI{239,5}{}$    & $\SI{5,146}{}$ \\
            $\SI{219,9}{}$    & $\SI{4,739}{}$ \\
            $\SI{199,2}{}$    & $\SI{4,324}{}$ \\
            $\SI{180,6}{}$    & $\SI{3,950}{}$ \\
            $\SI{160,7}{}$    & $\SI{3,576}{}$ \\
            $\SI{139,7}{}$    & $\SI{3,136}{}$ \\
            $\SI{119,8}{}$    & $\SI{2,762}{}$ \\
            \hline
            \label{tab:Messwerte-Zeitdiffernz-Abstand-Wasser}
        \end{tabular}
    \end{center}
\end{table}

\begin{table}[h!]
    \begin{center}
        \caption{Berechnete Brechzahlen für verschiedene Materialien}
        \begin{tabular}{cccc}
            \hline
            Messung            & Wasser ($\SI{1}{m}$)
            Mittelwert         & $\SI{3,591}{}$
            Standardabweichung & $\SI{0,015}{}$
            \hline
            \label{tab:Ergebnisse-Brechzahlen}
        \end{tabular}
    \end{center}
\end{table}


\subsection{Plexiglas}

\begin{table}[h!]
    \begin{center}
        \caption{Messungen der Phasenverschiebung für verschiedene Abstände mit kurzem Plexiglasstab, 8cm}
        \begin{tabular}{cc}
            \hline
            Abstand in $\SI{}{cm}$ & Zeitdifferenz in $\SI{}{\mu s}$ \\
            \hline
            $\SI{159,3}{}$    & $\SI{2,881}{}$ \\
            $\SI{140,0}{}$    & $\SI{2,508}{}$ \\
            $\SI{119,9}{}$    & $\SI{2,161}{}$ \\
            $\SI{100,3}{}$    & $\SI{1,772}{}$ \\
            $\SI{80,3}{}$     & $\SI{1,374}{}$ \\
            $\SI{60,6}{}$     & $\SI{9,672}{}$ \\
            $\SI{40,2}{}$     & $\SI{5,776}{}$ \\
            $\SI{20,2}{}$     & $\SI{1,872}{}$ \\
            \hline
            \label{tab:Messwerte-Zeitdiffernz-Abstand-Plexi-kurz}
        \end{tabular}
    \end{center}
\end{table}

\begin{table}[h!]
    \begin{center}
        \caption{Messungen der Phasenverschiebung für verschiedene Abstände mit langem Plexiglas Stab, 30cm}
        \begin{tabular}{cc}
            \hline
            Abstand in $\SI{}{cm}$ & Zeitdifferenz in $\SI{}{\mu s}$ \\
            \hline
            $\SI{180,5}{}$    & $\SI{3,55}{}$ \\
            $\SI{160,4}{}$    & $\SI{3,135}{}$ \\
            $\SI{139,5}{}$    & $\SI{2,762}{}$ \\
            $\SI{119,7}{}$    & $\SI{2,348}{}$ \\
            $\SI{99,2}{}$     & $\SI{1,941}{}$ \\
            $\SI{80,4}{}$     & $\SI{1,602}{}$ \\
            $\SI{60,0}{}$     & $\SI{1,196}{}$ \\
            $\SI{40,8}{}$     & $\SI{0,8063}{}$ \\
            \hline
            \label{tab:Messwerte-Zeitdiffernz-Abstand-Plexi-lang}
        \end{tabular}
    \end{center}
\end{table}

\begin{table}[h!]
    \begin{center}
        \caption{Berechnete Brechzahlen für Plexiglas}
        \begin{tabular}{cccc}
            \hline
            Messung            & Plexiglas ($\SI{8}{cm}$) & Plexiglas ($\SI{30}{cm}$)  \\
            Mittelwert         & $\SI{4,103}{}$        & \multicolumn{2}{c}{\SI{1,507}{}} \\
            Standardabweichung & $\SI{0,015}{}$       & \multicolumn{2}{c}{\SI{0,09}{}} \\
            \hline
            \label{tab:Ergebnisse-Brechzahlen}
        \end{tabular}
    \end{center}
\end{table}

\section{Bestimmung der Brechzahl mittels Lissajous-Figuren}

Lissajous-Figuren sind Kurvengraphen, die aus zwei senkrecht zueinander stehenden, harmonischen Schwingungen bestehen. Sie entstehen, wenn man das Oszilloskop im X-Y-Modus betreibt, wobei die Spannung der LED an der X-Achse und die Spannung am Lichtempfänger auf der Y-Achse aufgetragen wird. Wenn man die Phasenverschiebung bei einem festen Abstand nun so einstellt, dass die kreisförmige Lissajous-Figur zu einer Geraden zusammenfällt, lässt sich durch verschieben der LED eine zweite Position finden, an der das passiert.
Zur Berechnung der Lichtgeschwindkeit mit dieser Methode misst man die Strecke, um die die LED verschoben wurde.

Es ergibt sich mit einer Verschiebung um 2,618m:

\begin{equation}
    c = 2 \cdot \lambda  \cdot f = 4 \cdot f \cdot \Delta s = 3,9 \cdot 10^8\,.
\end{equation}