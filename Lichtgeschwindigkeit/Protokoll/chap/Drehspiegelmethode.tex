\section{Vorbereitung auf den Versuch}

In diesem Versuch geht es darum, die Größe der Lichtgeschwindigkeit mithilfe der Drehspiegelmethode zu bestimmen. Dafür wird ein Laserstrahl über einen Strahlenteiler geschickt, wobei dieser dahinter auf einen Drehspiegel trifft, der sich mit regulierbarer Frequenz dreht. Der Strahl durchquert eine Linse und trifft, nach einer weiteren Umlenken durch einen Spiegel, senkrecht auf einen Endspiegel. Der Strahlengang kehrt sich nun um, wobei sich der Drehspiegel in der Zwischenzeit um einen Winkel $\delta$ weitergedreht hat. Vom Strahlteiler wird dieser nun auf den Schirm geschickt.\\
Gemessen wird nun der Abstand a, also der Abstand zu dem Punkt, wo der Strahl bei ruhendem Drehspiegel auftreffen würde. Dieser wird auf der Millimeterskala auf dem Schirm für unterschiedliche Frequenzen gemessen.\\
Da der Versuch schon aufgebaut wurde, sind einige Werte, die für die Auswertung wichtig sind, schon gegeben. Diese sind in Tabelle \ref{tab:Drehspiegel vorgegebene Werte} aufgeführt.

\begin{table}[h]
    \centering
    \caption{Gegebene Werte Versuchsaufbau Drehspiegel}
    \begin{tabular}{c c c}
    \hline
    Abstand & Symbol & Wert \\
    \hline
    Laser-Drehspiegel Maximal & $d_{1max}$ & \SI{6.8}{\metre} \\
    Laser-Drehspiegel & $d_1$ & \SI{6.51}{\metre}\\
    Drehspiegel-Umlenkspiegel & $d_2$ & \SI{7.23}{\metre}\\
    Umlenkspiegel-Enspiegel & $d_3$ & \SI{6.59}{\metre}\\
    Brennweite Linse & $f$ & \SI{5}{\metre}\\
    
    \hline
    \end{tabular}
    \label{tab:Drehspiegel vorgegebene Werte}\\
    Fehler auf alle gegebenen Werte \SI{0.03}{\metre}.
\end{table}

Vorbereitend sollen nun die Positionen von Linse und Laseraustrittsöffnung berechnet werden. In diesem Versuchsaufbau wird eine Linse verwendet, damit der Strahl sowohl auf dem Dreh- als auch auf dem Enspiegel gebündelt wird und man somit ein scharfes Bild bekommt. Damit dies der Fall ist, muss der Abstand zwischen Drehspiegel und Linse gerade der Brennweite entsprechen. Um nun kann man sich die Gesetze der geometrischen Optik zu nutze machen \cite{2}. Man erhält durch Anwendung der Linsenleichung: 

\begin{equation}
    \frac{1}{f} = \frac{1}{b} + \frac{1}{g} = \frac{1}{d_1 + f} + \frac{1}{d_2 - f + d_3}
\end{equation}

Diese Formel wird nun nach $d_1$ umgestellt und man erhält:

\begin{equation}
    d_1 = \frac{f^2}{d_2+d_3-2f} \approx \SI{6.544}{\metre}
\end{equation}

Dieser Wert dient nur zur Überprüfung des voreingestellten Werts und dieser kann somit, da es nur eine kleine Abweichung gibt, übernommen werden.

Aus dem Abstand $a$ kann man nun auf die Lichtgeschwindigkeit schließen, die Formel dafür wird im folgenden hergeleitet:\\
Der Abstand a ist abhängig vom Winkel $\delta$, um den sich der Drehspiegel dreht, während das Licht den skizzierten Weg zurücklegt. Man erhält:

\begin{equation} \label{a}
    a = d_1 \tan{2\delta} \approx 2\delta d_1
\end{equation}

Man erhält nun weiter für die Zeitdifferenz $\Delta t$, die der Laserstrahl braucht:

\begin{equation}
    \Delta t = \frac{\delta}{\omega} = \frac{\delta}{2\pi f} \stackrel{(\ref{a})}{=} \frac{a}{4\pi f d_1}
\end{equation}

Und für die Streckendifferenz $\Delta s$:

\begin{equation}
    \Delta s = 2(d_2 + d_1)
\end{equation}

Der Wert für die Lichtgeschwindigkeit ergibt sich nun aus der Differenz von $\Delta s$ und $\Delta t$:

\begin{equation} \label{Drehspiegelmethode Formel}
    c = \frac{\Delta s}{\delta t} = \frac{8\pi f d_1 ( d_2 + d_3)}{a}
    \Rightarrow a = \frac{8 \pi f d_1 (d_2 + d_3)}{c}
\end{equation}

Es soll zusätzlich noch der erwartete Effekt berechnet werden, dafür werden die gegebenen Werte, zusammen mit dem Literaturwert für die Lichtgeschwindigkeit $c$ und einer Frequenz von $f = \SI{500}{\Hz}$ in die Formel \ref{Drehspiegelmethode Formel} eingesetzt und man erhält $a \approx \SI{0.00377}{\metre} = \SI{3.77}{\milli\metre}$.

\section{Justierung der Apparatur und Messung}


\section{Auswertung}