\section{Reduzierte Pendellänge}

Es soll vorbereitend die reduzierte Pendellänge berechnet werden, also die Länge, die ein mathematisches Pendel haben würde, damit es mit gleicher Schwingungsdauer schwingen würde, wie das zu beschreibende physikalische Pendel. Das physikalische bzw. Reversionspendel, für das die reduzierte Pendellänge berechnet werden soll, besteht aus einem zylindrischen, an einem Ende drehbar aufgehängte Stab der Länge L.
Nach der Grundgleichung für Drehbewegungen gilt:

\begin{equation} \label{Grundgleichung Drehbewegung}
    M = I \cdot \ddot{\varphi}
\end{equation}

wobei $M$ das rücktreibende Drehmoment, $I$ das Trägheitsmoment des Pendels und $varphi$ die Winkelbeschleunigung ist. Für das rücktreibende Drehmoment gilt: $M = -m \cdot g \cdot s \cdot \sin{\varphi}$ mit der Kleinwinkelnäherung $\sin{\varphi}$ folgt nach Einsetzen in Gleichung \ref{Grundgleichung Drehbewegung}:

\begin{equation} \label{Schwingungsgleichung eines physikalischen Pendels}
\ddot{\varphi} + \omega^2\varphi = 0
\Rightarrow \varphi = e^{i \omega t} \text{, mit } \omega = \sqrt{\frac{mgs}{I}}
\end{equation}

Mit dem Ergebnis für die Kreisfrequenz $\omega$ lässt sich nun die Persiodendauer $T$ berechnen:

\begin{equation}
    T = \frac{2\pi}{\omega} = 2 \pi \sqrt{\frac{I}{mgl}}
\end{equation}

Das Trägheitsmoment eines Zylinders mit einer homogenen Masseverteilung ist gegeben durch $I_Z = \frac{1}{3} ml^2$. Der Schwerpunkt des Zylinders liegt bei $s = l/2$. Daraus folgt für die Periodendauer des physikalischen Pendels:

\begin{equation}
    T = 2 \pi \sqrt{\frac{I}{mgs}} = 2 \pi \sqrt{\frac{\frac{1}{3} l^2}{mg\frac{l}{2}}} = 2 \pi \sqrt{\frac{2l}{3g}}
\end{equation}

Durch einen Vergleich mit der Periodendauer eines mathematische Pendels: $$T_m = 2 \pi \sqrt{\frac{l_r}{g}}$$ lässt sich erkennen, dass die reduzierte Pendellänge $l_r = \frac{2}{3}l$ beträgt.

Weiter soll vorbereitend gezeigt werden, dass eine zusätzlich angebrauchte Masse bei $l_r$  keine Veränderung der Periodendauer hervorruft. Also gilt für $I^\prime = m^\prime (\frac{2}{3}l)^2$ und für $s^\prime = \frac{2}{3}l$

\begin{equation}
    T^\prime = 2 \pi \sqrt{\frac{I + I^\prime}{mgs+m^\prime g s^\prime }} = 2\pi \sqrt{ \frac{ (\frac{m}{2} + \frac{2}{3} m^\prime) \frac{2}{3}l^2} { (\frac{m}{2} + \frac{2}{3}m^\prime)g l  }} \equiv T
\end{equation}

Mit dieser Erkenntnis lässt sich argumentieren, dass die Klauen, an denen das Pendel aufgehängt wird, eine zu vernachlässigende Änderung der Periodendauer mit sich bringen.

\section{Bestimmung der Fallbeschleunigung g mit Hilfe des Reversionspendels}

