\section{Aufbau und Vorbereitung}

Im letzten Versuch wird das Verhalten gekoppelter Pendel untersucht. Der Aufbau besteht aus zwei mit einer Feder gekoppelten Stabpendeln, an denen jeweils eine Massescheibe angebracht ist.
Bevor man die Pendel koppelt, muss jedoch gesichert sein, dass die Pendel sich auch identisch verhalten.
Dazu werden zunächst der Dreh- und Massenpunkt festgelegt und dann die Periodendauern je Pendel gemessen und nach linearer Regression verglichen.

\begin{align}
    l &= \SI{80}{cm} \\
    L &= \SI{111}{cm} \\
    a &= \SI{31}{cm} \\
    m_{\text{Massescheibe}} &= \SI{1221}{g}
\end{align}

\begin{table}[h!]
    \begin{center}
        \caption{Zeiten für unterschiedliche Periodendauern beider Pendel}
        \begin{tabular}{ccc}
            \hline
            Anzahl Perioden & Zeit linkes Pendel in $\SI{}{s}$ & Zeit rechtes Pendel in $\SI{}{s}$ \\
            \hline
            3  & $\SI{5,51}{}$ & $\SI{5,36}{}$ \\
            5  & $\SI{9,09}{}$ & $\SI{8,84}{}$  \\
            7  & $\SI{12,53}{}$ & $\SI{12,51}{}$  \\
            9  & $\SI{15,98}{}$ & $\SI{16,08}{}$ \\
            \hline
            \label{tab:Schwingungen-Pendel-einzeln}
        \end{tabular}
    \end{center}
\end{table}

\section{Fundamentalschwingung}

Zur Analyse der Fundamentalschwingung werden die Pendel nun mit einer geeigneten Feder verbunden und die Periodendauer von Hand per Stoppuhr gemessen.
Geeignet bedeutet hier, dass die Federkonstante nicht zu groß sein sollte, da die Pendel sonst so schnell schwingen, dass die Zeit nicht mehr akkurat gestoppt werden kann.
Es wurde für zwei Koppellängen gemessen, wobei "Koppellänge" den Abstand ziwschen Drehpunkt und Federbefestigung bezeichnet.

\section{Auswertung}

Die Auswertung geschieht erneut über Regression.

[PLOT]
[PLOT]
[TABELLE]

Andersweitige Bestimmung der Federkonstante

Im diesem Versuchteil soll die Federkonstante der vorher verwendeten Feder jeweils einmal statisch und dynamisch bestimmt werden.
Für die statische Methode wird die Feder durch das Anhängen unterschiedlich schwerer Gewichte unterschiedlich stark aus ihrer Ruheposition ausgelenkt.

[Tabelle]

Für die dynamische Methode wird ein Gewicht von blubba an die Feder gehängt und dann in Schwinung versetzt. Wie in blubba wurde die Schwingungsdauer per Stoppuhr erfasst.

[Tabelle]

Auswertung

Die Auswertung passiert erneut über Regression.


[Plot]

[Plot]

Statisch ergibt sich mit [Gleichung] eine Federkonstante von

Dynamisch ergibt sich mit [Gleichung] eine Federkonstante von

Schwingungs- und Schwebungsdauer gekoppelter Oszillatoren

Nun werden noch die Schwingungs- und die Schwebungsdauer des gekoppelten Pendels aus blubba gemessen.
Dazu wird ein Pendel ausgelenkt, während das andere in Ruhe verbleibt.

[Tabelle]

Zur Auswertung wurde erneut Regression verwendet, das Ergebnis für die Schwingungsdauer lautet blubba, das für die Schwebungsdauer blobba.